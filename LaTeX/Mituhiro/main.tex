\documentclass{exam}

\usepackage[portuguese]{babel}
\usepackage[most]{tcolorbox}
\usepackage{amssymb}
\usepackage{amsthm}
\renewcommand\qedsymbol{QED}

\newtcbtheorem{resposta}{Resposta}{colback=cyan!15!white, colframe=cyan!90!black}{th}

\title{%
    Matemática Discreta I \\
    \large ACH2013 -- Turma 94 \\
    1ª Prova \\
    Universidade de São Paulo \\
    Escola de Artes, Ciências e Futebol \\
}
\author{Eu mesmo}

\begin{document}
\maketitle
\begin{questions}
    \question[1,5] Considere as seguintes proposições:
    \begin{parts}
        \part $\left(p \leftrightarrow \left(\neg q \lor r\right)\right) \rightarrow \left(\neg p \rightarrow q\right)$
        \part $(p \rightarrow (q \lor r)) \lor (p \rightarrow q)$ \\
        Mostre que somente uma delas é uma tautologia.
    \end{parts}
    \begin{resposta}{}{}
        \[\left(p \leftrightarrow \left(\neg q \lor r\right)\right) \rightarrow \left(\neg p \rightarrow q\right)\]
        \[\equiv \neg (\neg (p \lor (\neg q \lor r)) \lor (p \land (\neg q \lor r))) \lor (p \lor q)\]
        \[\equiv ((p \lor \neg q \lor r) \land \neg((p \land \neg q) \lor (p \land r))) \lor p \lor q\]
        \[(p \rightarrow (q \lor r)) \lor (p \rightarrow q) \equiv (\neg p \lor (q \lor r)) \lor (\neg p \lor q)\]
        \[\equiv \neg p \lor q \lor r \lor \neg p \lor q\]
        \[\equiv \neg p \lor q \lor r\]
    \end{resposta}
    \question[2,0] Considere as seguintes premissas ``Não está chovendo hoje ou Mario tem um guarda-chuva.''
    ``Mario não tem um guarda-chuva ou ele não pegou chuva.'' ``Está chovendo hoje ou Mario não pegou chuva''
    e mostre que isso leva à conclusão que ``Mario não pegou chuva.''
    \begin{resposta}{}{}
        $p \equiv$ ``Está chovendo hoje'', $q \equiv$ ``Mario tem um guarda-chuva'',
        $r \equiv$ ``Mario pegou chuva''
        \[(\neg p \lor q) \land (\neg q \lor \neg r) \land (p \lor \neg r)\]
        \[\equiv (p \rightarrow q) \land (q \rightarrow r) \land (\neg p \rightarrow \neg r)\]
    \end{resposta}
\end{questions}
\end{document}
