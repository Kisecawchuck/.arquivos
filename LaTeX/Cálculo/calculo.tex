\documentclass{report}

\usepackage[portuguese]{babel}
\usepackage{hyperref, fancyhdr}
\usepackage{amsmath, amssymb}
\usepackage{yhmath}
\usepackage{tikz, xcolor}
\usepackage[most]{tcolorbox}

\setlength{\headheight}{13.07225pt}
\lhead{\thepage}
\cfoot{}
\pagestyle{fancy}

\newcommand{\D}{\mathop{}\!{d}}
\newcommand{\sen}{\operatorname{sen}}
\newcommand{\arcsen}{\operatorname{arcsen}}
\newcommand{\arco}{\operatorname{arco}}
\newcommand{\raio}{\operatorname{raio}}
\newcommand{\area}{\operatorname{\acute area}}
\renewcommand{\arraystretch}{1.6}

\newtheorem{exercicio}{Exercício}
\newtheorem{teorema}{Teorema}[section]

% Definindo o ambiente libretext
\tcbset{
    libretext/.style={
        enhanced,
        sharp corners,
        attach boxed title to top right={yshift*=-\tcboxedtitleheight+12pt, xshift=-\tcboxedtitlewidth+55pt},
        colframe=#1!30!black,
        colback=#1!5,
        colbacktitle=#1!5,
        toprule=0pt, rightrule=0pt, bottomrule=0pt,
        coltitle=black,
        fonttitle=\itshape,
    }
}

% Definindo o ambiente de demonstração
\newtcbtheorem{demo}{Demonstração}{
    libretext=cyan,
}{th}

\begin{document}
\tableofcontents

\chapter{Primitivas e derivadas}
\section{Definição e propriedades}
\label{sec:1.1}
A derivada pode ser pensada como um ``operador'' $D$ que para uma função f
suficientemente regular, ou seja, diferenciável, atribui uma outra função,
sua derivada, na forma
\[D\colon f \rightarrow f'.\]
Por exemplo, usando os resultado em [4, exemplo 1.1.6], sabe-se que
\[D\sen = \cos, \quad D\cos=-\sen.\]
Dado que a derivada opera nesse ``universo'' de funções, dada uma função $f$
em geral, cabe perguntar quando uma tal função resulta da forma $f = Dg$ para
alguma $g$ apropriada. Em outras palavras, do ponto de vista algébrico, esta
questão consiste em analisar o domínio do operador inverso $D^{-1}f=g$.
Por questões históricas, denota-se
\[\int{f}=g \Longleftrightarrow g'=f.\]
Em tal caso, se diz que $g$ é uma \textbf{primitiva} de $f$. Os termos
\textbf{antiderivada} e \textbf{integral indefinida} também costumam ser
usados na literatura com o mesmo significado.

Para uma função diferenciável $f$, o leitor não deveria ter nenhuma dificuldade
em perceber que
\[\int{f'}=f.\]
Caso contrário, talvez não tenha entendido corretamente a definição de primitiva,
em cujo caso seria recomendável revisar o conteúdo desde o começo da presente
seção.

Uma primeira propriedade que deve ser notada consiste em que a atribuição
definica pela primitiva contém um traço de ambiguidadem no sentido que uma
mesma função pode possuir várias primitivas diferentes. No entanto, a diferença
entre duas primitivas quaisquer não pode ser arbitrária. Com maior precisão,
tem-se
\[\int{f}=g \wedge \int{f} = h \Longrightarrow g = h + c.\]
para alguma constante $c \in \mathbb{R}$. Este resultado é um corolário do
teorema do valor médio, vide [\autoref{sec:2.2}].

Uma outra propriedade importante segue observando que, sendo a derivada um
operador $linear$, ou seja,
\[D(cf + g)=(cf+g)'=cf'+g'=cDf+Dg\]
tal linearidade é herdada pela primitiva, ou seja,
\[\int{(cf +g)}=c\int{f}+\int{g}.\]
A verificação desta última identidade fica a cargo do leitor, mais como um
exercício no uso da notação do que uma prova de perícia técnica.

Finalmente, uma outra questão que o leitor não deveria ter dificuldade
em se convencer é dada pela constatação de que uma maneira particularmente
simples de obter uma tabela de primitivas consiste em ler uma tabela
de derivadas no sentido contrário. A tabela 1.1 apresenta uma coletânea
minimalista de primitivas.

\begin{table}[h]
    \centering
    \begin{tabular}{ | c | c | }
        \hline\hline
        Função & Primitiva \\
        \hline
        $x^{n}$ & $\frac{x^{n+1}}{n+1}$ \\
        \hline
        $\cos$ & $\sen$ \\
        \hline
        $\sen$ & $-\cos$ \\
        \hline
        $\exp$ & $\exp$ \\
        \hline
        $\frac{1}{x}$ & $\log$ \\
        \hline
        $\frac{1}{1 + x^2}$ & $\arctan$ \\
        \hline\hline
    \end{tabular}
    \caption{Uma coletânea minimalista de primitivas.}
\end{table}

\section{Uma identidade notável}
\label{sec:1.2}
Nesta seção introdutória, o tratamento das séries é informal. No entanto,
tais manipulações podem ser formalizadas rigorosamente usando polinômios de
Taylor, incluindo estimativas adequadas do resto. Esta análise será adiada pelo
momento. O leitor interessado pode consultar as seções 6.3 e 6.4.

A identidade
\[\frac{1}{1 + x}=1-x+x^2-x^3+\cdot\cdot\cdot\]
como também
\[\frac{1}{1 - x}=1+x+x^2+x^3+\cdot\cdot\cdot\]
foram ambas estabelecidas em [\autoref{sec:6}]. Observe que
\[\int{\frac{1}{1+x}}\D{x} = \log{(1 + x)}, \quad \int{x^n}\D{x}=\frac{x^{n+1}}{n+1}.\]
Tomando a primitiva em ambos termos da primeira identidade acima, tem-se
\[\log{(1 + x)}=x-\frac{x^2}{2}+\frac{x^3}{3}-\frac{x^4}{4}+\cdot\cdot\cdot\]
Analogamente, observando que
\[\int{\frac{1}{1-x}}\D{x}=-\log{(1-x)}\]
e tomando a primitiva em ambos termos da segunda identidade, tem-se
\[-\log{(1-x)}=x+\frac{x^2}{2}+\frac{x^3}{3}+\frac{x^4}{4}+\cdot\cdot\cdot\]
Combinando ambas séries para o logaritmo, resulta
\[\log\frac{1+x}{1-x}=\log{(1+x)}-\log{(1-x)}\]
\[=(x + x) + \left(\frac{x^2}{2}-\frac{x^2}{2}\right)+\left(\frac{x^3}{3}+\frac{x^3}{3}\right)+\left(\frac{x^4}{4}-\frac{x^4}{4}\right)+\cdot\cdot\cdot\]
\[=2\left(x+\frac{x^3}{3}+\frac{x^5}{5}+\frac{x^7}{7}+\cdot\cdot\cdot\right)\]
de onde segue que
\[\frac{1}{2}\log{\frac{1+x}{1-x}}=x+\frac{x^3}{3}+\frac{x^5}{5}+\frac{x^7}{7}+\cdot\cdot\cdot\]

Por outro lado, a identidade
\[\frac{1}{1+x^2}=1-x^2+x^4-x^6+\cdot\cdot\cdot\]
também foi estabelecida em [\autoref{sec:6}]. Agora, observando que
\[\int{\frac{1}{1+x^2}}\D{x}=\arctan{x}\]
e tomando a primitiva em ambos termos desta identidade, tem-se
\[\arctan{x}=x-\frac{x^3}{3}+\frac{x^5}{5}-\frac{x^7}{7+\cdot\cdot\cdot}\]
Seja $i$ um símbolo tal que $i^2$ = $i \cdot i$ pode ser substituído por $-1$
em qualquer fórmula. Observe que
\[i=i\]
\[i^2=i\cdot i=-1\]
\[i^3=i^2\cdot i=-1\cdot i = -i\]
\[i^4=i^3 \cdot i = -i \cdot i = -(-1) = 1\]
Logo, tem-se
\[i^5=i^4 \cdot i = 1 \cdot i = i\]
e a partir daqui as sucessivas potências repetem-se periodicamente a cada
quatro, por exemplo:
\[i^5=i, \quad i^6=-1, \quad i^7=-i, \quad i^8=1\]
Usando essa tabela de potências do símbolo $i$ tem-se
\[\arctan{ix}=ix-\frac{(ix)^3}{3}+\frac{(ix)^5}{5}-\frac{(ix)^7}{7}+\cdot\cdot\cdot\]
\[=ix+i\frac{x^3}{3}+i\frac{x^5}{5}+i\frac{x^7}{7}+\cdot\cdot\cdot\]
\[=i\left(x+\frac{x^3}{3}+\frac{x^5}{5}+\frac{x^7}{7}+\cdot\cdot\cdot\right)\]
de onde segue que
\[-i\arctan{ix}=x+\frac{x^3}{3}+\frac{x^5}{5}+\frac{x^7}{7}+\cdot\cdot\cdot\]
Comparando com a identidade para o logaritmo estabelecida acima, obtém-se a
seguinte identidade notável
\[-i\arctan{ix}=\frac{1}{2}\log{\frac{1+x}{1-x}}\]
Esta identidade é equivalente à fórmula de Euler, como o leitor pode verificar
no exercício 1.5.1.

Para quem ficar incomodado com o fato da identidade acima ter sido estabelecida
através de manipulações informais de séries de potências, a fórmula de Euler
será obtida de maneira independente, e rigorosa, na \autoref{sec:8.4}.

Além disso, a identidade acima possui um equivalente que dispensa o uso
do símbolo $i$, segundo o resultado do exercício 1.6.2.

\section{Um exemplo revelador}
As primitivas foram introduzidas na \autoref{sec:1.1}. Uma aplicação é fornecida
na \autoref{sec:1.2} anterior, particulamente notável no sentido que o
resultado final apresentado equivale à fórmula de Euler.

No entanto, até agora não ficou clara a questão de, dada uma função arbitrária,
determina se possui ou não uma primitiva. Menos ainda foi dito sobre como
encontrar tal primitiva, caso a resposta da questão anterior seja afirmativa.

Um fator que complica ambas questões consiste no fato que muitas funções,
mesmo aquelas consideradas mais ``simples'', possuem primitivas que não
são tão elementares assim, ou cuja determinação não possui nada de trivial,
nem a resposta parece intuitiva. Por exemplo, o leitor pode verificar que
\[\int{\log{x}}\D{x}=x\log{x}-x\]
Para tanto, basta derivar o membro direito e verificar que o resultado
corresponde à função no membro esquerdo. No entanto, como pode ser determinada
essa primitiva em particular? Que tipo de cálculos devem ser desenvolvidos
com tal finalidade?

Colocando tais questões em um contexto mais amplo: Como surgem as primitivas?
Existe algum ``mecanismo'' capaz de gerar primitivas? A seguir é apresentado
um exemplo revelador para a questão da gênese das primitivas.

A figura 1.1 mostra um setor circular no primeiro quadrante do círculo unitário,
ou seja, de raio igual a 1. Observe que tal círculo é definido pela condição
\[S_1=\{(x,y) \in \mathbb{R}^2\colon x^2+y^2=1\}\]
de onde segue que
\[x^2+y^2=1 \Longrightarrow y^2=1-x^2 \Longrightarrow y = \pm\sqrt{1-x^2}\]
Em particular, o círculo unitário no semiplano $superior$ corresponde ao
gráfico da função
\[f(x)=\sqrt{1-x^2}\]

\begin{figure}
    \centering
    \includegraphics[width=0.6\textwidth, height=0.45\textheight]{circulo}
    \caption{Setor circular no círculo unitário.}
\end{figure}
\pagebreak

Seja $F(x)$ a área limitada pelo gráfico desta função $f$ e o eixo horizontal,
compreendida entre as abcissas $0$ e $x$. Observando a figura 1.1 tem-se
\[\area{\widehat{ACB}}=F(x)-\area\widehat{Ax_0B}\]
Denotando $a$ o arco subentendido pelo setor circular $\widehat{CAB}$, observe
que
\[x=\sen{a} \Longrightarrow a = \arcsen{x}\]
Além disso, tem-se
\[\area{\widehat{CAB}}=\frac{1}{2}(\arco)\cdot(\raio)=\frac{1}{2}a=\frac{1}{2}\arcsen{x}\]

\chapter{Calculando áreas}
\section{Aproximação por funções simples}
\begin{demo*}{}{}
    Uma prova no caso do intervalo $[0,1]$ consta em \autoref{sec:21.4}. A
    mesma demonstração pode ser adaptada para o intervalo $[0,b]$ com $b > 0$
    arbitrário como segue. Seja $\varepsilon > 0$ arbitrário. Dado que $[0,b]$ é
    compacto, pelo teorema \autoref{sec:21.4} segue que $f$ é uniformemente
    contínua. Portanto, existe $\delta > 0$ tal que
    \[|x-y| < \delta \Longrightarrow |f(x) - f(y)| < \varepsilon\]
    Considere uma partição do intervalo $[0,b]$ em $n$ subintervalos iguais,
    de comprimento $\frac{b}{n}$, onde $frac{b}{n} < \delta$. Seja $g$ definida como
    \[g(x)=f\left(\frac{b}{n}\left\lfloor{\frac{nx}{b}}\right\rfloor\right)\]
    Observe que para todo $x \in [0,b]$ existe algum $k \in \{0,1,...,n\}$ tal
    que $x$ pertence a $[\frac{bk}{n}, \frac{b(k+1)}{n}]$. Logo, tem-se
    \[x \in \left[\frac{bk}{n}, \frac{b(k + 1)}{n}\right] \Longrightarrow \frac{bk}{n} \leqslant x < \frac{b(k+1)}{n} \Longrightarrow k \leqslant \frac{nx}{b} < k + 1\]
    \[\Longrightarrow \frac{b}{n}\left\lfloor\frac{nx}{b}\right\rfloor=\frac{bk}{n}\]
    de onde segue que
    \[|f(x)-g(x)|=\left|f(x)-f\left(\frac{b\left\lfloor\frac{nx}{b}\right\rfloor}{n}\right)\right|=\left|f(x)-f\left(\frac{bk}{n}\right)\right| < \varepsilon\]
    pois
    \[x \in \left[\frac{bk}{n}, \frac{b(k+1)}{n}\right] \Longrightarrow \left|x-\frac{bk}{n}\right| < \frac{b}{n} < \delta\]
    Dado que as duas últimas duas desigualdades vigoram para todo $x \in [0,b]$
    segue que $||f-g|| < \varepsilon$.
\end{demo*}
Resulta como corolário que toda função $f$ contínua em $[0,b]$ pode ser
aproximada uniformemente em tal intervalo por uma sequência $\{g_n\}$ de
funções simples. Em outras palavras $||f-g_n|| \rightarrow 0$ quando $n$ tende
para infinito. Com efeito, basta usar o resultado anterior com $\varepsilon=\frac{1}{n}$,
para cada $n \in \mathbb{N}$.

Observe que área determinada pelo gráfico das funções simples $g_n$ pode ser
calculado somando a área de retângulos, como
\[\int_{0}^{b}{g_n}=\sum_{k=0}^{n-1}{\frac{b}{n}\cdot g_n\frac{bk}{n}}=\frac{b}{n}\sum_{k=0}^{n-1}{g_n\frac{bk}{n}}\]
\section{Funções trigonométricas}
\label{sec:2.2}
Considere a função $f(x)=\sen{(x)}$ no intervalo $[0, b]$ com $0 \leq b \leq \frac{\pi}{2}$.
Em tal caso, tem-se
\[\frac{b}{n}\sum_{k=0}^{n-1}{f\left(\frac{bk}{n}\right)}=\frac{b}{n}\sum_{k=0}^{n-1}{\sen{\left(\frac{bk}{n}\right)}}\]
\section{Recapitulação}
\section{Epílogo: O perigo mora nos detalhes}

\chapter{Integral de Riemann}
\section{Definição de integral}
\section{Funções simples}
\section{Lema de Riemann-Lebesgue}
\section{Área do círculo unitário com trapézios}
\section[Estimativa do valor de Pi]{Estimativa do valor de $\pi$}
\section{Uma estimativa do fatorial}
\section{Crescimento sublinear do logaritmo}

\chapter{Propriedades da integral}
\section{Integrabilidade em subintervalos}
\section{Linearidade}
\section{Integrais e desigualdades}
\section{Produto de funções}

\chapter{Funções contínuas e integral}
\section{Integrabilidade das funções contínuas}
\section{Teorema do valor médio para integrais}
\section{Teorema fundamental do Cálculo}
\begin{demo*}{}{}
    \[Df=g\]
\end{demo*}
\section{Convergência uniforme e integrais}
Seja agora $f \colon [a, b] \rightarrow \mathbb{R}$ contínua e considera uma
sequência $\{g_n\}$ de funções simples tal que
\[\lim_{n\rightarrow\infty}||f-g_n||=0\]
como no teorema 2.1.1, lembrando que
\[||f-g||=\sup{|f(x)-g(x)|\colon x \in [a,b]}.\]
\begin{teorema} Se $f$ é contínua em $[a,b]$, então
    \[\int_a^b{f}=\lim_{n\rightarrow\infty}\int_a^b{g_n}.\]
\end{teorema}

\chapter{Teorema fundamental do Cálculo}
\section{Continuidade da integral}
Teste
\label{sec:6}
\section{Primeiro teorema fundamental}
\section{Segundo teorema fundamental}

\section{Placeholder}
\label{sec:8.4}
\label{sec:21.4}
\end{document}
